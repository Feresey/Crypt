\documentclass[12pt]{article}

\usepackage[utf8x]{inputenc}
\usepackage{cmap} % для работы поиска кириллицы в pdf
\usepackage[T1, T2A]{fontenc}
\usepackage{fullpage}
\usepackage{multicol,multirow}
\usepackage{tabularx}
\usepackage{ulem}
\usepackage{listings}
\usepackage[english,russian]{babel}
\usepackage{tikz}
\usepackage{pgfplots}
\usepackage{ulem}
\usepackage{url}
\usepackage{hyperref}

\parindent=1cm
\linespread{1}
\pgfplotsset{compat=1.16}

\lstdefinestyle{custom}{belowcaptionskip=1\baselineskip,
  breaklines=true,
  frame=L,
  xleftmargin=\parindent,
  numbers=left,
  showstringspaces=false,
  basicstyle=\footnotesize\ttfamily,
  keywordstyle=\bfseries\color{green!40!black},
  commentstyle=\itshape\color{purple!40!black},
  identifierstyle=\color{blue},
  stringstyle=\color{orange},
}

% \lstset{escapechar=@,style=customc}

% \makeatletter

\newcommand{\se}[1]{\section*{\bf #1}}
\newcommand{\pa}[1]{\subsection*{\bf #1}}

\newcommand{\print}[2]{\smallbreak{\large  \bf  #1}
	\begin{verse}
	\input{#2}
	\end{verse}
}

\newcommand{\printPlain}[1]{\smallbreak{\large \bf  #1}
	{\scriptsize
		\setlength{\parindent}{0pt}
		\lstinputlisting[breaklines=true]{../#1}
	}
}

\renewcommand{\labelenumii}{\arabic{enumi}.\arabic{enumii}.}

\linespread{1.1}
\hfuzz=2pt
\vbadness10

\begin{document}

\section*{\centering Лабораторная работа №\,4 по курсу:\\ криптография}

Выполнил студент группы М8О-308Б-17 МАИ \,\, \textit{Милько Павел}.

\se{Задача}

Сравнить:

\begin{enumerate}
	\item Два осмысленных текста на естественном языке
	\item Осмысленный текст и текст из случайных букв.
	\item Осмысленный текст и текст из случайных слов.
	\item Два текста из случайных букв.
	\item Два текста из случайных слов.
\end{enumerate}

Как сравнивать: считать процент совпадения букв в сравниваемых текстах -- получить
дробное значение от 0 до 1 как результат деления количества совпадений на общее число
букв. Расписать подробно в отчёте алгоритм сравнения и приложить сравниваемые тексты в
отчёте хотя бы для одного запуска по всем пяти подпунктам. Осознать какие значения
получаются в этих пяти подпунктах. Привести свои соображения о том почему так
происходит.

Длина сравниваемых текстов должна совпадать. Привести соображения о том какой длины
текста должно быть достаточно для корректного сравнения.

\subsection*{Алгоритм сравнения}

Символы двух текстов сравниваются по их индексам относительно начала текста.
Необходимое отношение легко найти разделив количество совпадений на длину текста.

\subsection*{Входные данные}

В качестве примеров осмысленного текста я выбрал роман Жюля Верна -- ``20 000 лье под водой'' и
роман Джоан Роулинг -- ``Гарри Поттер и философский камень''.

\newpage

\print{20 000 лье под водой}{../testdata/humanity/jule_head.txt}
\print{Гарри Поттер и философский камень}{../testdata/humanity/harry_head.txt}

\subsection*{Отношения для отрывков осмысленных текстов}

\begin{table}[!th]
	\begin{tabular}{|l|l|l|}
		\hline
		\multicolumn{1}{|c|}{\begin{tabular}[c]{@{}c@{}}Количество\\ сравниваемых\\ символов\end{tabular}} & \multicolumn{1}{c|}{\begin{tabular}[c]{@{}c@{}}Смещение\\ относительно\\ начала файла\end{tabular}} & \multicolumn{1}{c|}{\begin{tabular}[c]{@{}c@{}}Отношение\\ совпавших символов\\ к общему количеству\end{tabular}} \\ \hline
200 & 0 & 0.085000\\ \hline
200 & 200 & 0.030000\\ \hline
500 & 0 & 0.066000\\ \hline
1500 & 1500 & 0.034667\\ \hline
10000 & 0 & 0.065400\\ \hline
346489 & 100500 & 0.050413\\ \hline
446989 & 0 & 0.064865\\ \hline
	\end{tabular}
\end{table}

Можно заметить что для первых 200 символов совпадение очень большое, хотя сами тексты сильно различаются.
Мне показалось это интересным и я добавил параметр смещения, чтобы изучить немного другие части файлов.

На следующих 200 символах совпадение оказалось более чем скромным - около 3\%. При сравнении более
больших кусков файлов получилось примерно 5.5\%-5.7\% абсолютного совпадения. Выглядит весьма неплохо
для абсолютно различных текстов.

\newpage

\subsubsection*{Сравнение осмысленного текста и рандомного набора символов}

\begin{table}[!th]
	\begin{tabular}{|l|l|l|}
		\hline
		\multicolumn{1}{|c|}{\begin{tabular}[c]{@{}c@{}}Количество\\ сравниваемых\\ символов\end{tabular}} & \multicolumn{1}{c|}{\begin{tabular}[c]{@{}c@{}}Смещение\\ относительно\\ начала файла\end{tabular}} & \multicolumn{1}{c|}{\begin{tabular}[c]{@{}c@{}}Отношение\\ совпавших символов\\ к общему количеству\end{tabular}} \\ \hline
200 & 0 & 0.010000\\ \hline
200 & 200 & 0.002500\\ \hline
500 & 0 & 0.010000\\ \hline
1500 & 1500 & 0.007667\\ \hline
10000 & 0 & 0.013900\\ \hline
346489 & 100500 & 0.009481\\ \hline
446989 & 0 & 0.012222\\ \hline
	\end{tabular}
\end{table}

\subsubsection*{Сравнение осмысленного текста и рандомного набора слов}

\begin{table}[!th]
	\begin{tabular}{|l|l|l|}
		\hline
		\multicolumn{1}{|c|}{\begin{tabular}[c]{@{}c@{}}Количество\\ сравниваемых\\ символов\end{tabular}} & \multicolumn{1}{c|}{\begin{tabular}[c]{@{}c@{}}Смещение\\ относительно\\ начала файла\end{tabular}} & \multicolumn{1}{c|}{\begin{tabular}[c]{@{}c@{}}Отношение\\ совпавших символов\\ к общему количеству\end{tabular}} \\ \hline
200 & 0 & 0.070000\\ \hline
200 & 200 & 0.037500\\ \hline
500 & 0 & 0.072000\\ \hline
1500 & 1500 & 0.026667\\ \hline
10000 & 0 & 0.056700\\ \hline
516782 & 100500 & 0.048897\\ \hline
617282 & 0 & 0.058341\\ \hline

	\end{tabular}
\end{table}

\newpage

\subsubsection*{Сравнение двух рандомных наборов слов}

\begin{table}[!ht]
	\begin{tabular}{|l|l|l|}
		\hline
		\multicolumn{1}{|c|}{\begin{tabular}[c]{@{}c@{}}Количество\\ сравниваемых\\ символов\end{tabular}} & \multicolumn{1}{c|}{\begin{tabular}[c]{@{}c@{}}Смещение\\ относительно\\ начала файла\end{tabular}} & \multicolumn{1}{c|}{\begin{tabular}[c]{@{}c@{}}Отношение\\ совпавших символов\\ к общему количеству\end{tabular}} \\ \hline
200 & 0 & 0.045000\\ \hline
200 & 200 & 0.040000\\ \hline
500 & 0 & 0.052000\\ \hline
1500 & 1500 & 0.032667\\ \hline
10000 & 0 & 0.056400\\ \hline
519501 & 100500 & 0.049477\\ \hline
620001 & 0 & 0.058902\\ \hline
	\end{tabular}
\end{table}

\subsubsection*{Сравнение двух рандомных наборов символов}

\begin{table}[!th]
	\begin{tabular}{|l|l|l|}
		\hline
		\multicolumn{1}{|c|}{\begin{tabular}[c]{@{}c@{}}Количество\\ сравниваемых\\ символов\end{tabular}} & \multicolumn{1}{c|}{\begin{tabular}[c]{@{}c@{}}Смещение\\ относительно\\ начала файла\end{tabular}} & \multicolumn{1}{c|}{\begin{tabular}[c]{@{}c@{}}Отношение\\ совпавших символов\\ к общему количеству\end{tabular}} \\ \hline
		200 & 0 & 0.035000\\ \hline
200 & 200 & 0.000000\\ \hline
500 & 0 & 0.016000\\ \hline
1500 & 1500 & 0.007667\\ \hline
10000 & 0 & 0.012400\\ \hline
519500 & 100500 & 0.010544\\ \hline
620000 & 0 & 0.012518\\ \hline
	\end{tabular}
\end{table}

По полученным данным видно что осмысленный текст и набор слов имеют почти такой же процент
совпадения как и 2 осмысленных текста.
Но и 2 набора слов так же имеют высокий процент совпадения, хотя никакой организации в них нет.

Обратная картина получается на случайных символах. Совпадение с реальным текстом чуть больше 1\%
Так что можно назвать это случайностью.
При сравнении двух наборов символов я первый раз получил 0\% совпадения, так что можно сказать что
никакого совпадения нет.

\subsection*{Выводы}

Я не ожидал увидесь совпадение в целых 5\%, ожидал около нуля. Сравнение отдельных слов так же дало 
высокий процент совпадения.

Такой высокий процент говорит о том что слова естественного языка намного более структурированы, чем
случайный набор символов.

Действительно, буквы используются в языке неравномерно. Те же `e' и `j' используются много чаще чем
`q' и `z'. Для осмысленных текстов, которые я использовал в лабораторной на букву `e' приходится
почти 10\% всех символов текста, а на ту же `z' 0.06\%.

Eсли воспользоваться знаниями о распределении букв и создать текст, по своей наполненности буквами
похожий на реальный, то процент совпадения будет приблизительно таким же как и при сравнении двух
текстов.

Но в более реальных случаях можно отличать полную белиберду от естественного языка. И для этого
достаточно всего лишь пары тысяч символов.

\end{document}

