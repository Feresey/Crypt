\documentclass[12pt]{article}

\usepackage[utf8x]{inputenc}
\usepackage{cmap} % для работы поиска кириллицы в pdf
\usepackage[T1, T2A]{fontenc}
\usepackage{fullpage}
\usepackage{multicol,multirow}
\usepackage{tabularx}
\usepackage{ulem}
\usepackage{listings}
\usepackage[english,russian]{babel}
\usepackage{tikz}
\usepackage{pgfplots}
\usepackage{ulem}
\usepackage{url}
\usepackage{hyperref}

\parindent=1cm
\linespread{1}
\pgfplotsset{compat=1.16}

\lstdefinestyle{custom}{belowcaptionskip=1\baselineskip,
  breaklines=true,
  frame=L,
  xleftmargin=\parindent,
  numbers=left,
  showstringspaces=false,
  basicstyle=\footnotesize\ttfamily,
  keywordstyle=\bfseries\color{green!40!black},
  commentstyle=\itshape\color{purple!40!black},
  identifierstyle=\color{blue},
  stringstyle=\color{orange},
}

% \lstset{escapechar=@,style=customc}

% \makeatletter

\newcommand{\se}[1]{\section*{\bf #1}}
\newcommand{\pa}[1]{\subsection*{\bf #1}}

\newcommand{\print}[2]{\smallbreak{\large  \bf  #1}
	\begin{verse}
	\input{#2}
	\end{verse}
}

\newcommand{\printPlain}[1]{\smallbreak{\large \bf  #1}
	{\scriptsize
		\setlength{\parindent}{0pt}
		\lstinputlisting[breaklines=true]{../#1}
	}
}

\renewcommand{\labelenumii}{\arabic{enumi}.\arabic{enumii}.}

\linespread{1.1}
\hfuzz=2pt
\vbadness10

\begin{document}

\section*{\centering Лабораторная работа №\,4 по курсу:\\ криптография}

Выполнил студент группы М8О-308Б-17 МАИ \,\, \textit{Милько Павел}.

\se{Задача}

Сравнить:

\begin{enumerate}
	\item Два осмысленных текста на естественном языке
	\item Осмысленный текст и текст из случайных букв.
	\item Осмысленный текст и текст из случайных слов.
	\item Два текста из случайных букв.
	\item Два текста из случайных слов.
\end{enumerate}

Как сравнивать: считать процент совпадения букв в сравниваемых текстах -- получить
дробное значение от 0 до 1 как результат деления количества совпадений на общее число
букв. Расписать подробно в отчёте алгоритм сравнения и приложить сравниваемые тексты в
отчёте хотя бы для одного запуска по всем пяти подпунктам. Осознать какие значения
получаются в этих пяти подпунктах. Привести свои соображения о том почему так
происходит.

Длина сравниваемых текстов должна совпадать. Привести соображения о том какой длины
текста должно быть достаточно для корректного сравнения.

\subsection*{Алгоритм сравнения}

Символы двух текстов сравниваются по их индексам относительно начала текста.
Необходимое отношение легко найти разделив количество совпадений на длину текста.

\subsection*{Входные данные}

В качестве примеров осмысленного текста я выбрал роман Жюля Верна -- ``20 000 лье под водой'' и 
роман Джоан Роулинг -- ``Гарри Поттер и философский камень''.

\newpage

\print{20 000 лье под водой}{../testdata/jule_head.txt}
\print{Гарри Поттер и философский камень}{../testdata/harry_head.txt}



\end{document}

