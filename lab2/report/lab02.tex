\documentclass[12pt]{article}

\usepackage[utf8x]{inputenc}
\usepackage{cmap} % для работы поиска кириллицы в pdf
\usepackage[T1, T2A]{fontenc}
\usepackage{fullpage}
\usepackage{multicol,multirow}
\usepackage{tabularx}
\usepackage{ulem}
\usepackage{listings} 
\usepackage[english,russian]{babel}
\usepackage{tikz}
\usepackage{pgfplots}
\usepackage{ulem}
\usepackage{url}
\usepackage{hyperref}

\parindent=1cm
\linespread{1}
\pgfplotsset{compat=1.16}
\newcommand{\se}[1]{\section*{\bf #1}}
\newcommand{\pa}[1]{\subsection*{\bf #1}}

\lstdefinestyle{custom}{belowcaptionskip=1\baselineskip,
  breaklines=true,
  frame=L,
  xleftmargin=\parindent,
  numbers=left,
  showstringspaces=false,
  basicstyle=\footnotesize\ttfamily,
  keywordstyle=\bfseries\color{green!40!black},
  commentstyle=\itshape\color{purple!40!black},
  identifierstyle=\color{blue},
  stringstyle=\color{orange},
}

% \lstset{escapechar=@,style=customc}

% \makeatletter

\newcommand{\print}[2]{\smallbreak{\large  \bf  #1}
	{\scriptsize
		\lstinputlisting[style=custom,language=#2]{../#1}
	}
}

\newcommand{\printPlain}[1]{\smallbreak{\large \bf  #1}
	{\scriptsize
		\setlength{\parindent}{0pt}
		\lstinputlisting[breaklines=true]{../#1}
	}
}

\renewcommand{\labelenumii}{\arabic{enumi}.\arabic{enumii}.}

\linespread{1.1}
\hfuzz=2pt
\vbadness10

\begin{document}

\section*{\centering Лабораторная работа №\,2 по курсу:\\ криптография}

Выполнил студент группы М8О-308Б-17 МАИ \,\, \textit{Милько Павел}.

\se{Задача}

\begin{enumerate}
    \itemСоздать пару OpenPGP-ключей, указав в сертификате свою почту. Создать её возможно, например, с помощью дополнения {\small Enigmail к почтовому клиенту thunderbird}, или из командной строки терминала ОС семейства linux.
    \item Установить связь с преподавателем, используя созданный ключ, следующим образом:
    \begin{enumerate}
        \item Прислать собеседнику от своего имени по электронной почте сообщение, во вложении которого поместить свой сертификат открытого ключа и сам открытый ключ (как правило, они умещаются в одном файле).
        \item Дождаться письма, в котором собеседник Вам пришлет сертификат своего открытого ключа.
        \item Выслать сообщение, зашифрованное на ключе собеседника.
        \item Дождаться ответного письма.
        \item Расшифровать ответное письмо своим закрытым ключом.
    \end{enumerate}
    \item Собрать подписи под своим сертификатом открытого ключа.
    \begin{enumerate}
        \item Получить сертификат открытого ключа одногруппника.
        \item Убедиться в том, что подписываемый Вами сертификат ключа принадлежит его владельцу --- путём сравнения отпечатка ключа или ключа целиком, по доверенным каналам связи.
        \item Подписать сертификат открытого ключа одногруппника.
        \item Передать подписанный Вами сертификат полученный в п.3.2 его владельцу, т.е.одногруппнику.
        \item Повторив п.3.0.-3.3., собрать 10 подписей одногруппников под своим сертификатом.
        \item Прислать преподавателю свой сертификат открытого ключа, с 10-ю или более подписями одногруппников.
    \end{enumerate}
\item Подписать сертификат открытого ключа преподавателя и выслать ему.
\end{enumerate}

\pa{Часть 1: генерация ключа}

\lstinline|gpg --generate-key|

{\scriptsize
\begin{lstlisting}[escapechar=|,breaklines=true]
gpg (GnuPG) 2.2.19; Copyright (C) 2019 Free Software Foundation, Inc.
This is free software: you are free to change and redistribute it.
There is NO WARRANTY, to the extent permitted by law.

Note: Use "gpg --full-generate-key" for a full featured key generation dialog.

GnuPG needs to construct a user ID to identify your key.

Real name: |Милько Павел Алексеевич|
Email address: p.milko1999@yandex.ru
You are using the 'iso-8859-1' character set.
You selected this USER-ID:
    "|Милько Павел Алексеевич| <p.milko1999@yandex.ru>"

Change (N)ame, (E)mail, or (O)kay/(Q)uit? O
We need to generate a lot of random bytes. It is a good idea to perform
some other action (type on the keyboard, move the mouse, utilize the
disks) during the prime generation; this gives the random number
generator a better chance to gain enough entropy.
We need to generate a lot of random bytes. It is a good idea to perform
some other action (type on the keyboard, move the mouse, utilize the
disks) during the prime generation; this gives the random number
generator a better chance to gain enough entropy.
gpg: key 4DF7496E16FD3CBC marked as ultimately trusted
gpg: revocation certificate stored as '/home/lol/.gnupg/openpgp-revocs.d/703FD5D038947C5B57CCE5A64DF7496E16FD3CBC.rev'
public and secret key created and signed.

pub   rsa2048 2020-03-10 [SC] [expires: 2022-03-10]
      703FD5D038947C5B57CCE5A64DF7496E16FD3CBC
uid                      |Милько Павел Алексеевич| <p.milko1999@yandex.ru>
sub   rsa2048 2020-03-10 [E] [expires: 2022-03-10]
\end{lstlisting}
}

\pa{Экспорт сертификата ключа в файл:}
\lstinline|gpg --armor --export --output key.asc p.milko1999@yandex.ru|
\printPlain{key.asc}

\pa{Часть 2: расшифровка полученного сообщения с помощью ключа преподавателя}
Зашифрованный файл:

\printPlain{t3/encrypted.asc}

\noindent \lstinline|gpg --decrypt --output decrypted.txt encrypted.asc|
\smallbreak

\noindent Расшифрованный файл:

\printPlain{t3/decrypted.txt}

Сначала я подумал, что мне по ошибке пришёл дамп http-запроса, но потом я заметил строчку ``\lstinline|Content-Transfer-Encoding: quoted-printable|''.

После перекодирования тела сообщения из этой кодировки я получил следующее:

``Зашифрованное сообщение 3.''

\pa{Часть 3: сбор подписей одногруппников}

В качестве пробного одногруппника был выбран Алексей Куликов. Мы обменялись открытыми ключами и подписали их друг другу.

Теперь мой ключ выглядит так:
{
\scriptsize
\begin{lstlisting}[escapechar=|,breaklines=true]
pub   rsa2048 2020-03-10 [SC] [expires: 2022-03-10]
      703FD5D038947C5B57CCE5A64DF7496E16FD3CBC
uid           [ultimate] |Милько Павел Алексеевич| <p.milko1999@yandex.ru>
sig 3        4DF7496E16FD3CBC 2020-03-10  |Милько Павел Алексеевич| <p.milko1999@yandex.ru>
sig 3        B104CF5F4D4DE318 2020-03-13  Alexey <kapitoshka.the.first@gmail.com>
sub   rsa2048 2020-03-10 [E] [expires: 2022-03-10]
sig          4DF7496E16FD3CBC 2020-03-10  |Милько Павел Алексеевич| <p.milko1999@yandex.ru>
\end{lstlisting}
}

\end{document}

