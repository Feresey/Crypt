\documentclass[12pt]{article}

\usepackage[utf8x]{inputenc}
\usepackage{cmap} % для работы поиска кириллицы в pdf
\usepackage[T1, T2A]{fontenc}
\usepackage{fullpage}
\usepackage{multicol,multirow}
\usepackage{tabularx}
\usepackage{ulem}
\usepackage{listings}
\usepackage[english,russian]{babel}
\usepackage{tikz}
\usepackage{pgfplots}
\usepackage{ulem}
\usepackage{url}
\usepackage{hyperref}

\parindent=1cm
\linespread{1}
\pgfplotsset{compat=1.16}

\lstdefinestyle{custom}{belowcaptionskip=1\baselineskip,
  breaklines=true,
  frame=L,
  xleftmargin=\parindent,
  numbers=left,
  showstringspaces=false,
  basicstyle=\footnotesize\ttfamily,
  keywordstyle=\bfseries\color{green!40!black},
  commentstyle=\itshape\color{purple!40!black},
  identifierstyle=\color{blue},
  stringstyle=\color{orange},
}

% \lstset{escapechar=@,style=customc}

% \makeatletter

\newcommand{\se}[1]{\section*{\bf #1}}
\newcommand{\pa}[1]{\subsection*{\bf #1}}

\newcommand{\print}[2]{\smallbreak{\large  \bf  #1}
	{\scriptsize
		\lstinputlisting[style=custom,language=#2]{../#1}
	}
}

\newcommand{\printPlain}[1]{\smallbreak{\large \bf  #1}
	{\scriptsize
		\setlength{\parindent}{0pt}
		\lstinputlisting[breaklines=true]{../#1}
	}
}

\renewcommand{\labelenumii}{\arabic{enumi}.\arabic{enumii}.}

\linespread{1.1}
\hfuzz=2pt
\vbadness10

\begin{document}

\section*{\centering Лабораторная работа №\,3 по курсу:\\ криптография}

Выполнил студент группы М8О-308Б-17 МАИ \,\, \textit{Милько Павел}.

\se{Задача}

\begin{enumerate}
	\item Строку в которой записано своё ФИО подать на вход в хеш-функцию ГОСТ Р 34.11-2012
	      (Стрибог). Младшие 4 бита выхода интерпретировать как число, которое в дальнейшем будет
	      номером варианта. Процесс выбора варианта требуется отразить в отчёте.
	\item Программно реализовать один из алгоритмов функции хеширования в соответствии с
	      номером варианта. Алгоритм содержит в себе несколько раундов.
	\item Модифицировать оригинальный алгоритм таким образом, чтобы количество раундов
	      было настраиваемым параметром программы. в этом случае новый алгоритм не будет
	      являться стандартом, но будет интересен для исследования.
	\item Применить подходы дифференциального криптоанализа к полученным алгоритмам с
	      разным числом раундов.
	\item Построить график зависимости количества раундов и возможности различения
	      отдельных бит при количестве раундов 1,2,3,4,5,... .
	\item Сделать выводы.
\end{enumerate}

\pa{Определение номера варианта}

\includegraphics[width=\linewidth]{var.png}

Вариант $7$ - keccak (sha3)

\pa{Реализация}

В качестве реализации я использовал как основу данный \href{https://github.com/ebfe/keccak}{репозиторий}.
Для внесения изменений скопировал файл keccak.go в локальный проект.

Чтобы удостовериться что в коде действительно алгоритм SHA-3 я решил свериться со
\href{https://nvlpubs.nist.gov/nistpubs/FIPS/NIST.FIPS.202.pdf}{стандартом}.
Там даже есть ссылка на примеры (25 страница), но увы пользоваться ими очень неудобно
потому что приведены вычисления для сообщений размером 0, 5, 30, 1600, 1605 и 1630 бит.

Из этого чудного многообразия есть возможность протестировать корректность только для
примеров длиной
\href{https://csrc.nist.gov/CSRC/media/Projects/Cryptographic-Standards-and-Guidelines/documents/examples/SHA3-256_Msg0.pdf}{0}
и
\href{https://csrc.nist.gov/CSRC/media/Projects/Cryptographic-Standards-and-Guidelines/documents/examples/SHA3-256_1600.pdf}{1600}
бит, так как эти числа кратны восьми и можно передавать их байтами.

Тест с пустым сообщением выдал такой же хеш как и в официальном примере. Но для примера в 1600 бит
одинакового результата не получилось.

По самим примерам нереально даже понять правильно ли мною скопированы входные данные,
поэтому для проверки корректности результата я воспользовался официальной библиотекой
\href{https://pkg.go.dev/golang.org/x/crypto/sha3?tab=doc}{sha3} для языка го и
дополнительно проверил корректность аналогичной библиотекой для питона.




\end{document}

