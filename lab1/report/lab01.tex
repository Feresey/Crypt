\documentclass[12pt]{article}

\usepackage[utf8x]{inputenc}
\usepackage[T1, T2A]{fontenc}
\usepackage{cmap} % для работы поиска кириллицы в pdf
\usepackage{fullpage}
\usepackage{multicol,multirow}
\usepackage{tabularx}
\usepackage{ulem}
\usepackage{listings} 
\usepackage[english,russian]{babel}
\usepackage{tikz}
\usepackage{pgfplots}
\usepackage{ulem}
\usepackage{url}
\usepackage{hyperref}
\usepackage{noindentafter}

\parindent=1cm
\linespread{1}
\pgfplotsset{compat=1.16}
\newcommand{\se}[1]{\section*{\bf #1}}

\lstdefinestyle{custom}{
  belowcaptionskip=1\baselineskip,
  breaklines=true,
  frame=L,
  xleftmargin=\parindent,
  numbers=left,
  showstringspaces=false,
  basicstyle=\footnotesize\ttfamily,
  keywordstyle=\bfseries\color{green!40!black},
  commentstyle=\itshape\color{purple!40!black},
  identifierstyle=\color{blue},
  stringstyle=\color{orange},
}

% \lstset{escapechar=@,style=customc}

\makeatletter

\newcommand{\print}[2]{
	\smallbreak
	{\large  \bf  #1}
	{
		\scriptsize
		\lstinputlisting[style=custom,language=#2]{../#1}
	}
}

\newcommand
{\printPlain}[1]{
	\smallbreak
	{\large  \bf  #1}
	{
		\scriptsize
		\setlength{\parindent}{0pt}
		\lstinputlisting[breaklines=true]{../#1}
	}
}

\linespread{1.1}
\hfuzz=2pt
\vbadness10

\begin{document}

\section*{\centering Лабораторная работа №\,1 по курсу:\\ криптография}

Выполнил студент группы М8О-308Б-17 МАИ \,\, \textit{Милько Павел}.


\se{Задача}
Разложить каждое из чисел n1 и n2 на нетривиальные сомножители.\\
Вариант 12.


\noindent $n_1:$\\
\input{../tests/test1.format}

\noindent $n_2:$\\
\input{../tests/test2.format}

\se{Решение}

Для поиска простых делителей я пытался использовать алгоритм факторизации Ферма.
Метод основан на поиске чисел $x$ и $y$, которые удовлетворяют соотношению $x^2-y^2=n$,
что приводится к разложению $(x-y)(x+y)=n$.
Так же уравнение равносильно следующему: $x^2-n=y^2$, то есть $x^2-n$ -- квадрат.
Для начала алгоритма выбирается наименьшее $x$, такое что $x^2>n$.
Для каждого значения последующего значения вычисляют $(x+k)^2-n$ и проверяют, не является ли это число точным квадратом.
Если оно является точным квадратом, то получено разложение: $n=x^2-y^2=(x-y)(x+y)$.
Так же можно проверить число на простоту: если один из сомножителей единица, то исходное число -- простое.
Метод работает быстро, если $n$ явлеется произведением двух близких к друг другу сомножителей.


Программа обрабатывала первое число более 12 часов, но так и не завершила вычисления.
Поэтому я воспользовался готовой реализацией метода решета числового поля -- \href{https://sourceforge.net/projects/msieve/}{msieve}.


\printPlain{msieve.log}

Получанные числа действительно являются сомножителями n1 и при перемножении получается исходное число.

Со вторым числом такой фокус не прошёл, потому что $msieve$ н обрабатывает числа длиннее 311 символов
(Второе число состоит из 464 символов)

По совету старшекурсников, сомножители второго числа я искал как НОД с числом другого варианта. Тут мой код заработал как надо.


\print{find.py}{python}

\printPlain{test.res.format}


\end{document}

